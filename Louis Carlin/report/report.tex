%\setlength{\parindent}{0pt}
\documentclass{article}
\usepackage{amsmath}
\usepackage{amssymb}
\usepackage{amsthm}
\usepackage{amsopn}
\usepackage{MnSymbol}

\DeclareMathOperator{\range}{range}
\DeclareMathOperator{\nulls}{null}
\DeclareMathOperator{\spanf}{span}

\newcommand{\Z}{\ensuremath{\mathbb{Z}}}
\newcommand{\C}{\ensuremath{\mathbb{C}}}
\newcommand{\R}{\ensuremath{\mathbb{R}}}
\newcommand{\F}{\ensuremath{\mathbb{F}}}
\newcommand{\N}{\ensuremath{\mathbb{N}}}
\newcommand{\norm}[1]{\left\lVert#1\right\rVert}
\newcommand*\mean[1]{\bar{#1}}

\usepackage[activate={true,nocompatibility},final,tracking=true,kerning=true,spacing=true,factor=1100,stretch=10,shrink=10]{microtype}

\newtheorem{lemma}{Lemma}
\newtheorem{prop}{Proposition}
\newtheorem*{prop*}{Proposition}
\newtheorem*{cor*}{Corollary}
\newcommand{\ud}{\, \mathrm{d}}

\author{Louis Carlin}
\title{Euclidean Domains in Lean}

\usepackage[pdftex]{hyperref}
\hypersetup{colorlinks,%
	    filecolor=black,%
	    citecolor=black,%
	    linkcolor=black}



\begin{document}
\maketitle
Collaboration and Citations:
\begin{itemize}
    \item
\end{itemize}
\newpage 

%TODO
%Do I call rings/EDs with the mathematical letters or lean letters?

\section{A quick explanation of Lean}

%how to tie type theory in?
\subsection{An overview of dependent type theory}

%tactics mode (probably don't bother talking about monads)
\subsection{Tactics}

%constructivism

\section{Euclidean domains}
% definitional nuances?


\section{The Euclidean Algorithm}
%some history?

The Euclidean Algorithm is one of the main motivations of the definition of Euclidean Domains. 
It takes any two elements $a,b$ of a Euclidean Domain $R$ and gives a ``greatest common divisor'' of $a$ and $b$. 
An element $d \in R$ is a greatest common divisor of $a$ and $b$ if $d \divides a$, $d \divides b$ and $\forall x \in R, x \divides a \and x \divides b \implies x \divides d$.
We common write $\gcd(a,b)$ to denote a particular greatest common divisor of $a$ and $b$.
However it is important to note that greatest common divisors are not necessarily unique. In fact they are never unique in our definition of a Euclidean domain): if $d$ is a gcd of a $a$ and $b$ in a ring, then so is its additive inverse $-d$.

\subsection{The First implementation}
% proofs were inside the function which made things extremly hard to debug

\subsection{The Second implementation}
%much simpler
%had to define an induction principle
%talk about



\end{document}
