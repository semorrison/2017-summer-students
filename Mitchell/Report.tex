\documentclass[runningheads,a4paper]{llncs}

\usepackage[utf8]{inputenc}

\usepackage[square,sort,comma,numbers,sectionbib]{natbib}
\bibliographystyle{apalike-fr}
\renewcommand{\bibname}{References}

\usepackage{amssymb}
\setcounter{tocdepth}{3}
\usepackage{graphicx}
\usepackage{amsopn}
\usepackage{enumitem}
\usepackage{amsmath}
\usepackage{todonotes}

\topmargin=0in
\evensidemargin=0.25in
\oddsidemargin=0.25in
\textwidth=6in
\textheight=9.0in
\headsep=0.25in

\linespread{1.1}

%Blackboard Bold
\newcommand{\C}{\mathbb{C}}
\newcommand{\Q}{\mathbb{Q}}
\newcommand{\R}{\mathbb{R}}
\newcommand{\Z}{\mathbb{Z}}
\newcommand{\F}{\mathbb{F}}
\newcommand{\N}{\mathbb{N}}

%Calligraphic
\newcommand{\cC}{\mathcal{C}}
\newcommand{\cF}{\mathcal{F}}
\newcommand{\cA}{\mathcal{A}}
\newcommand{\cB}{\mathcal{B}}
\newcommand{\cP}{\mathcal{P}}
\newcommand{\cL}{\mathcal{L}}

%Greek
\renewcommand{\a}{\alpha}
\newcommand{\z}{\zeta}
\renewcommand{\b}{\beta}
\newcommand{\g}{\gamma}
\newcommand{\e}{\varepsilon}
\renewcommand{\d}{\delta}
\renewcommand{\r}{\rho}
\renewcommand{\l}{\lambda}
\newcommand{\w}{\omega}
\newcommand{\n}{\eta}
\newcommand{\f}{\varphi}
\newcommand{\s}{\sigma}
\renewcommand{\t}{\tau}
\newcommand{\tr}{\tilde \rho}


%Sets and Logic
\newcommand{\sub}{\subseteq}
\newcommand{\imp}{\implies}
\newcommand{\x}{\times}
\DeclareMathOperator{\End}{End}
\DeclareMathOperator{\Hom}{Hom}
\DeclareMathOperator{\im}{im}
\renewcommand{\-}{\setminus}
\DeclareMathOperator{\id}{id}

%Linear Algebra
\newcommand{\bop}{\bigoplus}
%\newcommand{\tr}{\text{tr}}

%Representation Theory
\newcommand{\fsl}{\mathfrak{sl}}
\newcommand{\fg}{\mathfrak{g}}
\newcommand{\gl}{\mathfrak{gl}}

%Galois Theory
\DeclareMathOperator{\Aut}{Aut}
\DeclareMathOperator{\Perm}{Perm}
\DeclareMathOperator{\Gal}{Gal}


\newtheorem{lemma}[theorem]{Lemma}
\newtheorem{proposition}[theorem]{Proposition}
\newtheorem{corollary}[theorem]{Corollary}

\begin{document}

\mainmatter 

\title{Group Theory in Lean}

\titlerunning{Group Theory in Lean}

\author{Mitchell Rowett}

\institute{Interactive Theorem Proving}

\authorrunning{Mitchell Rowett}

\toctitle{Abstract}
\tocauthor{{}}

\maketitle

\medskip

\begingroup
\let\clearpage\relax
\tableofcontents
\endgroup

\section*{Abstract}

In this report, we describe a formalisation of elementary group theory in the proof assistant Lean. We begin with an introduction to interactive theorem proving, and to the Lean Theorem Prover in particular. We then detail the construction of this formalisation in Lean, culminating in a proof of the first isomorphism theorem.

\section{Introduction}
Computational formal verification of mathematical theorems comes in two main forms. Automated theorem proving focuses on proving assertions fully automatically, with little or no input from a user. Interactive theorem proving involves a user actively guiding the proof, and builds mathematical structures 

\section{Group Theory}

\section{Lean Theorem Prover}

\subsection{Types}

\subsection{Structures}

\subsection{Type Classes}

\subsection{Type Classes and Groups}

\section{Homomorphisms}

\section{Subgroups}

\section{Cosets}

\section{Quotient Groups}

\section{First Isomorphism Theorem}


% \begin{thebibliography}{1}

% \end{thebibliography}

\end{document}
